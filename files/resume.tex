% \newcommand{\tab}[1]{\hspace{.2667\textwidth}\rlap{#1}}
% \newcommand{\itab}[1]{\hspace{0em}\rlap{#1}}
% \newcommand{\bibTitle}[1]{``#1''}

\documentclass{resume} 
\usepackage[left=0.6in,top=0.6in,right=0.8in,bottom=0.6in]{geometry} 
\pagenumbering{arabic}
\usepackage{setspace}

\usepackage{fontawesome}
\usepackage{enumitem}
\usepackage{datetime2}
\usepackage{tabularx}
\usepackage{needspace}
\usepackage{placeins}

\newcommand{\smallspace}{\hspace{0.2em}}

\usepackage{fancyhdr}
\pagestyle{fancy}
\fancypagestyle{firstpage}
{
    \fancyhead[L]{}    
    \fancyhead[R]{\color{date} Last Updated on \today}
}
\fancypagestyle{reference}
{
    \fancyfoot[L] {}
    \fancyfoot[C] {}
    \fancyfoot[R] {\color{date} \thepage}
}
\fancyhead{}
\fancyfoot{}
\renewcommand{\headrulewidth}{0pt}
\renewcommand{\footrulewidth}{0pt}
% \fancyfoot[L]{\color{date} Hongtao Hao~·~Curriculum Vitae}
% \fancyfoot[C]{\color{date} https://hongtaoh.com/en/vitae/}
\fancyfoot[C] {\color{date} \thepage}
% 483d8b, 8000000
\usepackage[dvipsnames]{xcolor}
\definecolor{hypercolor}{HTML}{510080}
\definecolor{sectioncolor}{HTML}{510080}
\definecolor{date}{HTML}{666666} 
\usepackage{hyperref}
\hypersetup{
    colorlinks=true,
    urlcolor=hypercolor,
}

% \usepackage{mathpazo} 
% \usepackage{fontspec}
% \setmainfont[
% BoldFont = Source Sans Pro-{Semi-bold}]{Source Sans Pro}

% Font configuaration
\usepackage[T1]{fontenc}
\usepackage{lmodern}  % Modern sans-serif font
\usepackage[scale=0.9]{sourceserifpro}  % Modern serif font for body text
\usepackage[scale=0.9]{sourcecodepro}  % Modern monospace font

% Set default fonts
\renewcommand{\familydefault}{\sfdefault}  % Make sans-serif the default
\renewcommand{\rmdefault}{sourceserifpro}
\renewcommand{\ttdefault}{sourcecodepro}

% \setromanfont[Mapping={tex-text}, 
% 	Numbers={OldStyle},
% 	Ligatures={Common}]%{Minion Pro}
%     {Caladea}
% \setmainfont[
% BoldFont = Open Sans {Semi-bold}]{Open Sans}
% \setsansfont[
% BoldFont = Roboto-Medium]{Roboto}

% Caladea
\name{Mitch Gerhardt}
% \address{hhao@iu.edu \hspace{0.4em} | \hspace{0.4em} \url{www.hongtaoh.com}}
\address{\hspace{0.2em} \hspace{0.4em} {\small \faEnvelope} \hspace{0.1em} mitchg@vt.edu \hspace{0.4em}
{\large\faGithub} \href{https://github.com/mitchg10/}{mitchg10} \hspace{0.4em} 
{\large\faLinkedinSquare} \href{https://linkedin.com/in/mitchell-gerhardt}{\texttt{mitchell-gerhardt}}} 
\begin{document}
% \setstretch{1.1}
\fancyheadoffset[RO]{5em}
\fancyfootoffset[RO]{0cm}
\thispagestyle{firstpage}

%----------------------------------------------------------------------------------------
%	Research Interests 
%----------------------------------------------------------------------------------------
\begin{rSection}{Research Interests}
% Engineering Education
% Natural Language Processing
% Machine Learning
% Qualitative Research Methods
% Epistemic Cognition
% Collaborative Learning
% Software Engineering
% Educational Technology
% Automated Analysis
% Large Language Models
% Engineering Design
% Educational Assessment
% AI in Education
% Capstone Design
% Python Development
% Research Methods
% Academic Integrity
% Data Analysis
% STEM Pedagogy
% Knowledge Representation
% Cognitive Science
% Human-Computer Interaction
% Web Application Development
% Teaching Excellence
    {\textbf{Sociotechnical studies of generative AI adoption in workplace settings, ethnographic methods, science and technology studies (STS), epistemic cultures, computational approaches to qualitative research, symbolic interactionism, workplace learning and expertise}}
\end{rSection}



%----------------------------------------------------------------------------------------
%	EDUCATION 
%----------------------------------------------------------------------------------------

\begin{rSection}{Education}
%--copy and paste this region  if you need more--
{\textbf {Ph.D. in Engineering Education}}, Virginia Tech, GPA: 4.00/4.00 \hfill {May, 2027 (expected)} 
\begin{itemize}
    \item[] \vspace{-0.5em} \hspace{-1em} PhD Candidate (advanced to candidacy Fall 2025)
    \item[] \vspace{-0.5em} \hspace{-1em} Committee: Courtney Faber, Andrew Katz (Chair), Holly Matusovich, and Nicole Pitterson
    \item[] \vspace{-0.5em} \hspace{-1em} 2025 NSF GRFP Honorable Mention
    \item[] \vspace{-0.5em} \hspace{-1em} 2024 Google PhD Fellowship Nominee
    \item[] \vspace{-0.5em} \hspace{-1em} 2023-2024 Davenport Fellowship Awardee
\end{itemize}

{\textbf {M.S. in Computer Science}}, Virginia Tech, GPA: 3.95/4.00 \hfill {December, 2026 (expected)} 
\begin{itemize}
    \item[] \vspace{-0.5em} \hspace{-1em} Committee: Sara Hooshangi (Chair), Andrew Katz, Sanmay Das
  \item[] \vspace{-0.5em} \hspace{-1.0em} Coursework: \emph{Machine Learning}, \smallspace \emph{Natural Language Processing}, \smallspace \emph{Computer Education Research}, \smallspace \emph{AI Tools for Software Engineering}, \smallspace \emph{Applied Design and Assessment of Educational Environments in Engineering}
\end{itemize}

{\textbf {Graduate Certificate: Cognition and Education}}, Virginia Tech, GPA: 4.00/4.00 \hfill {May, 2025} 

{\textbf {B.S. in Electrical Engineering}}, Virginia Tech, GPA: 3.69/4.00 \hfill {Dec., 2020}
\begin{itemize}
    \item[] \vspace{-0.5em} \hspace{-1.0em} Graduated Magna Cum Laude and Dean's List 
    \item[] \vspace{-0.5em} \hspace{-1.0em} Pratt Engineering Scholarship, Donald and Madeline Stewart Scholarship, and Konrad-Steinmetz Scholarship for Engineering Leadership Awardee
    \item[] \vspace{-0.5em} \hspace{-1.0em} Hardware Team Lead for \href{https://eng.vt.edu/academics/student-teams/autodrive.html}{AutoDrive Design Team}
\end{itemize}
\end{rSection}

%--------------------------------------------------------------------------------
%    Publications
%-----------------------------------------------------------------------------------------------

% Book Chapters

\begin{rSection}{Book Chapters}
% \begingroup
% \raggedbottom
\begin{tabularx}{\textwidth}{p{2.5cm}X}
Under \newline Revision & \textbf{Leveraging Large Language Models in Engineering Education Research: Methods and Applications}\\
& \emph{2026 International Handbook of Engineering Education Research
Methods}\\
& \textbf{Mitchell Gerhardt}, Gabriella Coloyan Fleming, Siqing Wei, and Andrew Katz\\
& Expected publication: February 2027\\
\end{tabularx}
% \endgroup
\end{rSection}

% Journal

\begin{rSection}{Journal Publications}
% \begingroup
% \raggedbottom
\begin{tabularx}{\textwidth}{p{2.5cm}X}
Pending \newline Publication & \textbf{Collaborative (In)decision: A Preliminary Investigation of the Differences in Undergraduate Engineering Capstone Students' Collaborative Behaviors}\\
& \emph{International Journal of Engineering Education} - Special Issue for Capstone Design\\
& \textbf{Mitchell Gerhardt}, Mayar Madboly, Nicole Pitterson, Emily Dringenberg, and Benjamin Ahn\\[1em]

Submitted & \textbf{Improving Engineering Education GAI Qualitative Research Workflow Quality: Techniques and Documentation Strategies}\\
& \emph{Studied in Engineering Education} - Special Issue on GAI in Methods\\
& \textbf{Mitchell Gerhardt} and Andrew Katz  \\

\end{tabularx}

\needspace{4\baselineskip} % Ensure space for at least 4 lines, otherwise pagebreak

\begin{tabularx}{\textwidth}{p{2.5cm}X}

In \newline Preparation & \textbf{Decade-Long Analysis of Skills in Mechanical Engineering Job Advertisements 2010-2022}\\
& \emph{Targeting: ASME}\\
& \textbf{Mitchell Gerhardt}, Shawn Sun, Andrew Katz, David Knight, Jessica Deters, Maura Borrego, Riya Budhathoki, and Herman Ronald Clements III  \\[2em]

% \needspace{4\baselineskip} % Check if we need to page break

In \newline Preparation & \textbf{Addictive GAI Use Among STEM Graduate Students: Dependency-Based Technological Diffusion}\\
& \emph{Targeting: Computers \& Education: AI}\\
& \textbf{Mitchell Gerhardt} and Andrew Katz  \\[1em]

Nov, 2024 & \textbf{Using Generative Text Models to Create Qualitative Codebooks for Student Evaluations of Teaching} \hspace{2em [\href{https://doi.org/10.1177/16094069241293283}{Paper}]}\\
& \emph{International Journal of Qualitative Methods} - Volume 23\\
& Andrew Katz, \textbf{Mitchell Gerhardt}, and Michelle Soledad\\

\end{tabularx}
% \endgroup
\end{rSection}

\needspace{4\baselineskip} % Check if we need to page break

% Conference

\begin{rSection}{Peer-Reviewed Conference Publications}
\begingroup
\raggedright
\begin{tabularx}{\textwidth}{p{2.5cm}X}

Accepted & \textbf{It's like “X”: How Engineering Faculty Metaphors Construct (and Constrain) GAI Understanding in Engineering Education}\\
& \emph{2026 ASEE Annual Conference \& Exposition}\\
& \textbf{Mitchell Gerhardt}, Kylee Shiekh, Andrew Katz, and Benjamin Chaback \\[1em]

Accepted & \textbf{WIP: Unpacking Mechanical Engineering Students' Career Goals, Skill Development, and Perspectives on Industry}\\
& \emph{2026 Frontiers in Education Conference}\\
& Herman Ronald Clements III, \textbf{Mitchell Gerhardt}, Jessica Deters, Shawn Sun, David Knight, Maura Borrego, Andrew Katz, and Riya Budhathoki \\[2em]

% \needspace{6\baselineskip} % Check if we need to page break

Accepted & \textbf{Understanding the Master’s Engineering Workforce Landscape: Employer Demands and Student Goals}\\
& \emph{2026 ASEE Annual Conference \& Exposition}\\
& Herman Ronald Clements III, Riya Budhathoki, \textbf{Mitchell Gerhardt}, Jessica Deters, Shawn Sun, David Knight, Maura Borrego, and Andrew Katz \\[2em]

Under \newline Revision & \textbf{When Can GAI be Used Anyway? An Analysis of Engineering Faculty’s Generative AI Policies}\\
& \emph{2026 ASEE Annual Conference \& Exposition}\\
& Benjamin Chaback, \textbf{Mitchell Gerhardt}, Andrew Katz, and Kylee Shiekh \\[1em]

June, 2025 & \textbf{Automated Analysis of Knowledge Types in Computer Science Textbooks: A Natural Language Processing Approach to Understanding Epistemic Climate} \hspace{2em [\href{https://peer.asee.org/55491}{Paper}]} \\
& \emph{2025 ASEE Annual Conference \& Exposition} - Montreal, QC, Canada \\
& \textbf{Mitchell Gerhardt} and Andrew Katz \\[1em]

June, 2025 & \textbf{Beyond Calculations: Engineering Judgment as Epistemic Cognition in Engineering Education} \hspace{2em [\href{https://peer.asee.org/beyond-calculations-engineering-judgment-as-epistemic-cognition-in-engineering-education}{Paper}]} \\
& \emph{2025 ASEE Annual Conference \& Exposition} - Montreal, QC, Canada \\
& \textbf{Mitchell Gerhardt}, Michael Robinson, and Brian Faulkner \\[1em]

June, 2024 & \textbf{Reimagining Behavioral Analysis in Engineering Education: A Theoretical Exploration of Reasoned Action Approach} \hspace{2em [\href{https://peer.asee.org/47934}{Paper}]} \\
& \emph{2024 ASEE Annual Conference \& Exposition} - Portland, OR \\
& \textbf{Mitchell Gerhardt}, Nicole Pitterson, Emily Dringenberg, and Benjamin Ahn \\
\end{tabularx}

\needspace{4\baselineskip} % Check if we need to page break
\begin{tabularx}{\textwidth}{p{2.5cm}X}

June, 2024 & \textbf{Why do capstone students choose to perform behaviors? Differing prevalence in collaborative choices} \hspace{2em [\href{https://capstonedesigncommunity.org/sites/default/files/proceedings_papers/Gerhardt-CDC24.pdf}{Paper}]} \\
& \emph{Capstone Design Conference} - Knoxville, TN \\
& \textbf{Mitchell Gerhardt}, Nicole Pitterson, Emily Dringenberg, and Benjamin Ahn \\
% \needspace{4\baselineskip} % Check if we need to page break

\end{tabularx}
\endgroup
\end{rSection}

%----------------------------------------------------------------------------------------
%	RESEARCH
%----------------------------------------------------------------------------------------
\begin{rSection}{Research experience}

%--CAREER Research Assistant with Andrew Katz--
{\bf Research Assistant} \hfill {\ Aug., 2025 - Present }\\ 
{\href{https://enge.vt.edu/}{Department of Engineering Education}, Virginia Tech} \hfill {\ {Blacksburg, VA}}\\
\href{https://www.nsf.gov/awardsearch/showAward?AWD_ID=2339702&HistoricalAwards=false}{NSF Award \#2339702}: ``CAREER: Minds and Machines: Exploring Engineering Faculty Member Mental Models of Generative AI and Instructional Decisions"
\begin{cvitems}
    \item Examining sociotechnical dimensions of GAI adoption in STEM higher education workplace contexts using mental model theory and Theory of Planned Behavior (TPB)
    \item Conducting qualitative interviews with faculty to understand how perceptions of GAI shape instructional decisions, assessment practices, and workplace norms around GAI use
    \item Applying ethnomethodological and symbolic interactionist frameworks to analyze epistemic cultures and expertise recognition in AI-mediated educational environments
    \item Developing LLM-based qualitative workflows to scale aspects of qualitative data analysis and make them available to fellow researchers
\end{cvitems}

%--Workforce Development Research Assistant --
{\bf Research Assistant} \hfill {\ May., 2025 - Present }\\ 
{\href{https://enge.vt.edu/}{Department of Engineering Education}, Virginia Tech} \hfill {\ {Blacksburg, VA}}\\
Assisted \href{https://www.nsf.gov/awardsearch/showAward?AWD_ID=2433099&HistoricalAwards=false}{NSF Award \#2433099}: ``Collaborative Research: Research: The Engineering Master's Workforce: Leveraging NLP Techniques to Understand Employer Demands and Student Goals":
\begin{cvitems}
    \item Created a Python-based workflow to manage DuckDB-based database to analyze over 5 million jobs spanning 2007-2025 
    \item Implemented Jupyter notebooks to extract job postings' qualification information, tasks and responsibilities, skills and knowledge, and technologies required
    \item Currently applying open-source large language models (LLMs) to add flexibility and granularity to job posting extractions
\end{cvitems}

%--Research Assistant with Andrew Katz--
{\bf Research Assistant} \hfill {\ Jan., 2024 - July, 2024 }\\ 
{\href{https://enge.vt.edu/}{Department of Engineering Education}, Virginia Tech} \hfill {\ {Blacksburg, VA}}\\
Assisted \href{https://www.nsf.gov/awardsearch/showAward?AWD_ID=2300977&HistoricalAwards=false}{NSF Award \#2300977}: ``Design for Sustainability: How Mental Models of Social-Ecological Systems Shape Engineering Design Decisions":
\begin{cvitems}
    \item Developed a novel natural language processing (NLP) methodology for generating qualitative codebooks from large datasets (e.g., students' instructor evaluations)
    \item Co-authored a journal publication in the \emph{International Journal of Qualitative Methods} outlining the extract, embed, cluster, and summarize (EECS) workflow
    \item Contributed to ongoing refinement and improvement of the NLP-based methodology for analyzing qualitative data
    \item Applied open-source LLMs to preserve data security and privacy while maintaining analytical rigor
\end{cvitems}

\needspace{4\baselineskip} % Check if we need to page break

%--Research Assistant with Nicole Pitterson--
{\bf Research Assistant} \hfill {\ August, 2024 - Jan., 2024 }\\ 
{\href{https://enge.vt.edu/}{Department of Engineering Education}, Virginia Tech} \hfill {\ {Blacksburg, VA}}\\
Assisted \href{https://www.nsf.gov/awardsearch/showAward?AWD_ID=2217523&HistoricalAwards=false}{NSF Award \#2217523}: ``Collaborative Research: Collaboration in Engineering Student and Practitioner Teams: A Study of Beliefs about Effective Behaviors":
\begin{cvitems}
    \item Designed and implemented a pilot study to examine the frequency and prevalence of collaborative behaviors in senior engineering capstone teams
    \item Co-authored a conference paper for the 2024 Capstone Design Conference about why students choose certain collaborative behaviors
    \item Expanded the conference paper into an accepted journal article based on the preliminary findings from the project
    \item Developed foundation for future research on student collaborative decision-making within engineering design contexts
\end{cvitems}

{\bf Ethnographic Researcher} \hfill {\ Fall, 2025}\\
{\href{https://sts.cornell.edu/}{Cornell University Science \& Technology Studies}} \hfill {\ {Ithaca, NY}}\\
Independent ethnographic study of Science Studies Reading Group (SSRG) feedback practices
\begin{cvitems}
    \item Conducted 15+ weeks of participant observation examining how different feedback formats (facilitated dialogues, "ping pong" discussions) shape scholarly collaboration and expertise networks
    \item Produced 10,000-word analytical ethnography applying STS frameworks to understand epistemic cultures, knowledge production practices, and organizational dynamics in interdisciplinary research communities
    \item Analyzed conversational patterns, turn-taking behaviors, and collaborative meaning-making in academic workplace settings
    \item Working on methodology paper explicating study and decision-making for contributions to studying workplace knowledge practices in EER
\end{cvitems}

{\bf Principal Investigator, Graduate Student Survey Research} \hfill {\ May, 2025 - Present}\\
{Multi-institutional} \hfill {\ Ithaca, NY}\\
IRB-approved multi-institutional study examining STEM graduate students' GAI use patterns
\begin{cvitems}
    \item Designed and distributed Qualtrics survey across a few universities collecting 100+ responses examining nuanced patterns of GAI adoption beyond reductive "cheating" narratives
    \item Applying mixed methods approach combining descriptive statistics, correlation analyses, demographic studies, and planned qualitative interviews
    \item Investigating workplace socialization processes, community norms around GAI use, and shifting definitions of legitimate expertise in graduate education contexts
\end{cvitems}

\end{rSection}

%--------------------------------------------------------------------------------
%    PROFESSIONAL EXPERIENCE
%-----------------------------------------------------------------------------------------------

\begin{rSection}{Professional Experience}

%--Graduate Assistant at CETL--
{\bf Graduate Assistant to the Director for Special Programs} \hfill {\ Aug., 2024 - May, 2025 }\\ 
{\href{https://teaching.vt.edu/}{Center for Excellence in Teaching and Learning (CETL)}, Virginia Tech} \hfill {\ {Blacksburg, VA}}\\
Worked with Dr. Tiffany Shoop on various teaching and learning initiatives:
\begin{cvitems}
    \item Conducted mid-semester feedback sessions across various disciplines, gathering and analyzing student input to help faculty enhance teaching effectiveness and student engagement
    \item Developed and presented a faculty workshop on Interactive Lecturing with approximately 30 attendees, providing resources and strategies for enhancing student engagement in the classroom
    \item Co-led the implementation of the \href{https://cirtl.vt.edu/}{CIRTL} (Center for the Integration of Research, Teaching, and Learning) program at Virginia Tech, researching best practices and developing an institutional implementation strategy for Practitioner and Scholar CIRTL levels leveraging cross-campus resources and partnerships
    \item Helped organize the 17th annual Conference on Higher Education Pedagogy (\href{https://chep.teaching.vt.edu/}{CHEP}), reviewing submissions, coordinating sessions, and developing a custom Python application to automate post-conference booklet creation
    \item Systematically reviewed empirical research on Collaborative Online Interactive Learning (COIL) to assist in faculty training program development and institutional integration
    \item Created resources for faculty including presentations, research compilations, and implementation guides to support teaching excellence and innovation
\end{cvitems}

% \needspace{13\baselineskip} % Check if we need to page break

%--Global Product Development Engineering at GM--
{\bf Global Product Development Engineering} \hfill {\ June, 2021 - June, 2023 }\\ 
\emph{General Motors} \hfill {\ {Warren, MI}}\\
{Software Engineer - Software Defined Vehicle}
\begin{cvitems}
    \item Responsible for developing next-gen software infrastructures and technology for software-defined vehicles
    \item Key implementation role in mobile app development, in-vehicle interfaces, and back-end deployment
    \item Established and applied testing protocols including CI/CD pipelines, automated build processes, and testing mechanisms
\end{cvitems}
{Infotainment Technical Program Manager}
\begin{cvitems}
    \item Part of an integral team to plan, execute, and deliver leading-edge technology to our customers
    \item Worked with global partners, external suppliers, and internal teams to track, drive, support, and report vehicle readiness metrics
\end{cvitems}
{Complexity Optimization Engineer}
\begin{cvitems}
    \item Engaged in measuring, managing strategy, and reducing program hardware and software complexity
    \item Optimized tactics for efficient build combinations and part reuse across programs and vehicle subsystems
\end{cvitems}

%--Autonomy Systems Research Engineering at Caterpillar--
{\bf Autonomy Systems Research} \hfill {\ May, 2019 - Dec., 2020 }\\ 
\emph{Caterpillar Inc.} \hfill {\ {Peoria, IL and Blacksburg, VA}}\\
{Software Engineer I}
\begin{cvitems}
    \item Implemented a control interface for autonomy simulation platform using ROS, teleoperation control, and robot modeling software
    \item Validated and integrated sensor data on large-scale machinery into algorithms for mapping and navigation via complex data structures
    \item Developed CI/CD protocols and workflows for software distribution using Docker, ROS, and the C++ testing framework
\end{cvitems}

\end{rSection}

%--------------------------------------------------------------------------------
%    GRANTS & AWARDS
%-----------------------------------------------------------------------------------------------
\begin{rSection}{Grants \& Awards}

{\bf Co-Principal Investigator} \hfill {\ Fall, 2025}\\
\textbf{Bridging the Conversational AI Gap: Synthetic Dataset Generation for Engineering Education Dialogue}\\
NSF CCSS Seed Grant (\$4,500), Cornell University - 10-month project
\begin{cvitems}
    \item Developing novel methodology for creating synthetic speech datasets preserving conversational phenomena critical to workplace learning research (overlapping speech, strategic pauses, collaborative turn-taking)
    \item Fine-tuning open-source ASR models using 500-1,000 multi-speaker dialogue samples generated via ElevenLabs API to enable large-scale analysis of engineering classroom interactions
    \item Applying synthetic data to study feedback literacy in design studios and facilitated dialogues about research group culture and inclusion
    \item Publicly archiving all synthetic audio, generation parameters, and evaluation results in Cornell CSSR Data \& Reproduction Archive for reproducible research across institutions
    \item Positions research program for larger NSF/NIH proposals by demonstrating technical feasibility of computational methods in engineering education research
\end{cvitems}

\needspace{10\baselineskip} % Check if we need to page break

{\bf Co-Facilitator} \hfill {\ Fall, 2025}\\
\textbf{Fulbright Brazil International Collaboration}\\
Virginia Tech and Brazilian Universities Partnership
\begin{cvitems}
    \item Selected by Fulbright Brazil with advisor and VT colleagues for intensive 3-day faculty development program examining GAI adoption in higher education workplace contexts
    \item Presented to 80+ engineering faculty from dozens of Brazilian universities on sociotechnical dimensions of GAI integration: technical fundamentals, assessment transformation, ethics, institutional policy development, faculty-student relationship dynamics
    \item Served as technical expert and instructional designer facilitating cross-cultural dialogue about workplace technology adoption and organizational change
\end{cvitems}

{\bf Nominee} \hfill {\ 2025}\\
\textbf{Google PhD Fellowship in Human-Computer Interaction}\\
Nominated by Virginia Tech for national fellowship

\needspace{3\baselineskip} % Check if we need to page break

{\bf Honorable Mention} \hfill {\ 2025}\\
\textbf{NSF Graduate Research Fellowship Program}\\
National Science Foundation

\end{rSection}

%--------------------------------------------------------------------------------
%    PROJECTS
%-----------------------------------------------------------------------------------------------
\begin{rSection}{Projects}

% Ducky AI Assistant
{\bf Ducky}{\hspace{2em}} \hfill {Fall, 2025} \\
AI-powered software developer assistant built with Streamlit providing intelligent development support through specialized features for CS 5740: AI Tools for Software Engineers
\begin{cvitems}
    \item Developed ReAct-pattern agent system with sandboxed tool execution capabilities, enabling GAI agents to autonomously perform file operations, code editing, and multi-step task completion with security through path validation
    \item Implemented dual interface architecture: rich web interface with Monaco editor integration and CLI interface, supporting persona-based agent behaviors across software development roles (planning, development, QA, operations)
    \item Integrated RAG (Retrieval Augmented Generation) using local sentence-transformers for offline-capable document queries and comprehensive LLM integration with OpenAI-compatible APIs supporting function calling and streaming responses
    \item Built with modern Python tooling including uv for dependency management, demonstrating full-stack development skills including async programming, state management, UI/UX design, and secure tool execution patterns
\end{cvitems}

% Between the Lines Bookstore
{\bf Between the Lines}{\hspace{2em}} \hfill {Fall, 2025} \\
React + TypeScript bookstore e-commerce application developed for CS 5244: Web App Development
\begin{cvitems}
    \item Built full-stack e-commerce site using React + TypeScript with Vite, evolving from Figma and static HTML/CSS into a complete bookstore application with shopping cart, category browsing, and checkout functionality
    \item Developed Java servlet backend (MitchBookstoreTransact) running on Tomcat to handle categories, books, and order processing with RESTful API design
    \item Implemented state management using React Context providers (CategoryContext, CartContext, OrderDetailsProvider) with useReducer pattern for cart and order state, including persistent cart state via localStorage
    \item Created component-based architecture with React Router v6 for navigation across six main routes: home, category pages, cart, checkout, and order confirmation
    \item Integrated Formik/Yup for form validation with custom validators for phone numbers and credit cards, ensuring robust client-side validation throughout checkout flow
\end{cvitems}

% College of Engineering Course Collection Tool
{\bf Graduate Courses Analysis Tool}{\hspace{2em} [\href{https://course-extractor-rior.onrender.com/}{Web}], [\href{https://github.com/mitchg10/course-extractor}{Code}]} \hfill {Nov., 2023 - May, 2025} \\
Full-stack web application for College of Engineering Associate Dean for graduate enrollment patterns
\begin{cvitems}
    \item Developed tool extracting structured course data from PDF timetables, identifying underenrolled graduate courses for strategic enrollment analysis and resource allocation decisions
    \item Technology Stack: React/Tailwind frontend, FastAPI backend, PyMuPDF for PDF processing, AWS S3 for production storage
    \item Implemented automated ETL pipeline for processing university course enrollment data at scale with cross-listing identification and data enrichment via VT Timetable API
    \item Containerized application using Docker for reproducible deployment and maintenance
\end{cvitems}

%--Epistemic Knowledge Analysis--
{\bf Epistemic Climate Analysis in CS Education}{\hspace{2em} [\href{https://github.com/mitchg10/NLP_FinalProject}{Code}]} \hfill {Sept., 2024 - June, 2025} \\
Sociotechnical analysis of knowledge representation in computer science textbooks using STS frameworks and computational methods
\begin{cvitems}
    \item Developed novel methodology analyzing epistemic climate in CS textbooks using large language models, examining how different knowledge types shape students' understanding of legitimate expertise and disciplinary identity
    \item Created comprehensive coding scheme for 12 knowledge types (procedural, conceptual, historical, etc.) grounded in engineering education research and STS literature on epistemic cultures
    \item Implemented retrieval-augmented generation approach for large-scale textbook analysis, demonstrating how computational methods can reveal patterns in how disciplines present and value different forms of knowledge
    \item Generated insights about knowledge hierarchies in undergraduate CS education, revealing sociotechnical dimensions of how textbooks as material artifacts shape epistemic thinking
    \item Published findings at ASEE 2025 Annual Conference
\end{cvitems}

\end{rSection}

%--------------------------------------------------------------------------------
%    SERVICE
%-----------------------------------------------------------------------------------------------

\begin{rSection}{Service}

{\bf Associate Chair, Panelist, \& Facilitated Discussion Mediator}\\
\emph{Virginia Tech Graduate Honor System} \hfill {\ Aug., 2023 - Present}
\begin{itemize}
    \item \vspace{-0.5em} Serve as a trained facilitator for academic integrity discussions between faculty and students, mediating resolution of potential honor code violations through structured dialogue \vspace{-0.3em}
    \item  \vspace{-0.5em} Participate in Review Panels that evaluate evidence and determine outcomes in academic integrity cases, upholding university standards while ensuring fair process \vspace{-0.3em}
    \item  \vspace{-0.5em} Help educate the graduate community on academic integrity standards, ethical research practices, and honor system processes \vspace{-0.3em}
    \item  \vspace{-0.5em} Collaborate with diverse stakeholders including faculty, administrators, and students across disciplines to resolve cases annually \vspace{-0.3em}
\end{itemize} 

{\bf Graduate Student Search Committee Member}\\
\emph{Department of Graduate and Professional Studies} \hfill {\ Summer, 2025}
\begin{itemize}
    \item[] \vspace{-0.5em} \hspace{-1em} Served on faculty search committee for GHS chair position, evaluating candidates and participating in selection process
\end{itemize}

{\bf Mentor, NSF Graduate Research Fellowship Program Applications}\\
\emph{Cornell University} \hfill {\ Fall, 2025}
\begin{itemize}
    \item[] \vspace{-0.5em} \hspace{-1em} Provided application review, feedback, and mentorship to Cornell graduate students applying for NSF GRFP
\end{itemize}

{\bf Graduate Ambassador}\\
\emph{Department of Engineering Education} \hfill {\ Aug., 2024 - May, 2025}

{\textbf {Reviewer}, College of Engineering Torgersen Research Excellence Award} \hfill {May, 2025} \\[0.5ex]
{\textbf {Reviewer}, ASEE Annual Conference} \hfill {May 2024, 2025, 2026} \\[0.5ex]
{\textbf {Reviewer}, International Journal of Qualitative Methods} \hfill {Nov., 2024 \& Sept., 2025} \\[0.5ex]
{\textbf {Reviewer}, Capstone Design Conference} \hfill {May, 2024} \\[0.5ex]
{\textbf {President Emeritus}, Hillel at Virginia Tech} \hfill {Dec., 2018 - Dec., 2020}
\end{rSection}

\needspace{4\baselineskip} % Check if we need to page break

%--------------------------------------------------------------------------------
%    Invited Talks/Presentations
%-----------------------------------------------------------------------------------------------
\begin{rSection}{Invited Talks, Workshops, and Presentations} 

{\bf Workshop}{, \emph{Virginia Tech Graduate Instructors}} {\hfill (planned) March, 2026}
\begin{itemize}
    \item \vspace{-0.5em} \hspace{-1em} Initiated and lead a three-student group to organize a workshop for Virginia Tech graduate instructors about GAI course policies
    \item \vspace{-0.5em} \hspace{-1em} Addressing a critical need through participatory instruction and activities that promote GAI literacy
    \item \vspace{-0.5em} \hspace{-1em} Received IRB approval to study the workshop's implementation and effectiveness, with plans to publish results  
    \item \vspace{-0.5em} \hspace{-1em} Leveraging networks and existing partnerships with the Graduate Honor System (GHS), the Center for Excellence in Teaching and Learning (CETL), the center for Technology-Enhanced Learning and Online Strategies (TLOS), and departmental leadership to host and market the event.
\end{itemize}

\needspace{5\baselineskip} % Check if we need to page break

{\bf Guest Lecturer}{, \emph{GRAD 5004: Graduate Teaching Assistant Workshop }} {\hfill Feb., 2026} \\
\emph{Virginia Tech (Dr. Kevin Eager)}\\
"Scholarly Ethics" presentation to in-coming GTAs at Virginia Tech, addressing topics like academic and professional integrity, the VT Undergraduate and Graduate Honor Systems, GTA integrity responsibilities, GAI use, and contemporary literature about cheating.  

{\bf Guest Lecturer}{, \emph{ECE Graduate Student Seminar}} {\hfill Dec., 2025} \\
\emph{University of Pittsburgh}\\
"AI Research Ethics in Engineering Education" co-presented with Andrew Katz to ECE graduate students examining ethical dimensions of AI research in educational contexts
% \vspace{-0.5em}

{\bf Guest Lecturer}{, \emph{ME 397/379M Qualitative Research Methods}} {\hfill Nov., 2025} \\
\emph{University of Texas at Austin (Dr. Maura Borrego)}\\
"Large Language Models in Qualitative Engineering Education Research: Technical Methods and Methodological Considerations"
\begin{itemize}
    \item \vspace{-0.5em} \hspace{-1em} Examined human vs. machine pattern recognition, current GAI approaches for qualitative research, researcher positionality, translational challenges, and shifting definitions of "intelligence"
    \item \vspace{-0.5em} \hspace{-1em} Addressed methodological questions about computational approaches to qualitative work and implications for research practice
    \item \vspace{-0.5em} \hspace{-1em} Instructor feedback: "Students were honestly blown away...for many of them it was the highlight of the semester"
\end{itemize}
% \vspace{-0.5em}

{\bf Invited Presentations}{, \emph{Cornell DBER Research Group (2 presentations)}} {\hfill Oct., 2025} \\
\emph{Cornell University}\\
Presented technical background of large language models, AI history, and sociotechnical applications in engineering education research; demonstrated how computational methods intersect with critical perspectives on workplace learning and expertise
% \vspace{-0.5em}

{\bf Conference Presentation}{, \emph{2025 ASEE Annual Conference}} {\hfill June, 2025}
\begin{itemize}
    \item \vspace{-0.5em} \hspace{-1em} "Automated Analysis of Knowledge Types in Computer Science Textbooks: A Natural Language Processing Approach to Understanding Epistemic Climate" - exploring the use of LLMs to evaluate types of knowledge described in computer science textbooks
    \item \vspace{-0.5em} \hspace{-1em} "Beyond Calculations: Engineering Judgment as Epistemic Cognition in Engineering Education" - arguing for greater psychological interrogation of "engineering judgment" 
\end{itemize}
% \vspace{-0.5em}

% \vspace{-0.5em}

% {Oct., 2025\hspace{1em}}{\bf Attendee}{, \emph{Trevor Pinch Memorial Lecture Series}} {\hfill Ithaca, NY} \\
% Cornell University Science \& Technology Studies Department\\
% Participated in memorial series for pioneering STS scholar, engaging with questions about social construction of technology and ethnographic methods in workplace studies
% \vspace{-0.5em}

{\bf Invited Student}{, \emph{Board of Visitors}} {\hfill May, 2025} \\
\emph{Virginia Tech}\\
Discussed research on GAI adoption in higher education and ongoing projects examining workplace technology integration with university leadership
% \vspace{-0.5em}

{\bf Panelist}{, \emph{2024 Capstone Design Conference}} {\hfill June, 2024} \\
Panel discussion examining teamwork dynamics and collaborative practices in capstone engineering workplace contexts
% \vspace{-0.5em}

{\bf Conference Presentation}{, \emph{2024 ASEE Annual Conference}} {\hfill June, 2024} \\
"Reimagining Behavioral Analysis in Engineering Education: A Theoretical Exploration of Reasoned Action Approach" - examining decision-making in engineering student workplace teams
% \vspace{-0.5em}

{\bf Conference Poster}{, \emph{2024 Capstone Design Conference}} {\hfill June, 2024} \\
"Why do capstone students choose to perform behaviors? Differing prevalence in collaborative choices"
% \vspace{-0.5em}

{\bf Panelist}{, \emph{Virginia Tech Interdisciplinary Capstone (IDC) course}} {\hfill Oct., 2023} \\
Discussion of engineering industry workplace practices and career pathways with capstone students
\vspace{-0.5em}

\end{rSection}

%--------------------------------------------------------------------------------
%    TEACHING EXPERIENCE
%-----------------------------------------------------------------------------------------------
\needspace{10\baselineskip} % Check if we need to page break

\begin{rSection}{Teaching Experience}

% Big brothers big sisters
{\bf Big Brothers Big Sisters of Metropolitan Detroit} \hfill {\ {Detroit, MI}}\\
Academic Tutor \hfill {\ Aug., 2021 - June, 2023}
\begin{itemize}
  \item \vspace{-0.5em} Helped create and support mission for one-on-one mentoring sessions and academic assistance \vspace{-0.3em}
  \item \vspace{-0.5em} Exposure to COVID-19 instruction methods and challenges thereof \vspace{-0.3em}
  % \item \vspace{-0.5em} Conviction to enable success regardless of socioeconomic background or previous academic performance \vspace{-0.3em}
\end{itemize} 

\end{rSection}

%----------------------------------------------------------------------------------------
%	TECHNICAL SKILLS
%----------------------------------------------------------------------------------------
\begin{rSection}{Technical Skills}

{\bf Qualitative Methods:} Ethnography, participant observation, semi-structured interviews, grounded theory, symbolic interactionism, ethnomethodology, thematic analysis, discourse and conversational analysis, content analysis\\
{\bf Theoretical Frameworks:} Engineering expertise, sociology of expertise, Science \& Technology Studies (STS), epistemic cultures, epistemic thinking, social construction of technology, engineering studies, academic integrity and cheating, human-computer interactions (HCI)\\
{\bf Computational Methods:} Natural language processing, large language models (LLMs), prompt and context engineering, LLMs for qualitative research, retrieval-augmented generation (RAG), agent-based systems, computational text analysis\\
{\bf Programming \& Tools:} Python, JavaScript, C++, SQL, React, FastAPI, Docker, Git, Jupyter notebooks, PyTorch, MCP, scikit-learn, HuggingFace Transformers, AI-based programming systems \\
{\bf Mixed Methods:} Survey design (Qualtrics), descriptive statistics, correlation analysis, triangulation of qualitative and quantitative data\\
{\bf Data Management:} DuckDB, MySQL, AWS S3, Azure, version control, reproducible research practices

\end{rSection}

% \newpage
\thispagestyle{reference}

%%%%%%%%%%%%%%%%% REFERENCES %%%%%%%%%%%%%%%%%%%%%%%%%%
% The reference section has links to your references' websites and email addresses.

% \noindent \begin{tabular}{@{} l l l}
%  \Large{References} & \href{https://mediaschool.indiana.edu/people/profile.html?p=nicomart}{Nicole Martins} & \href{http://yongyeol.com/}{Yong-Yeol Ahn} \\
%  & The Media School &  Luddy School of Informatics, Computing, and Engineering  \\
%  & Indiana University Bloomington &  Indiana University Bloomington \\
%  & \small{\href{mailto:nicomart@indiana.edu}{nicomart@indiana.edu},+1\,(812)\,855-7720} & \small{\href{mailto:yyahn@iu.edu}{yyahn@iu.edu},+1\,(812)\,856-2920} \\
% && \\
%  & \href{https://publichealth.indiana.edu/research/faculty-directory/profile.html?user=seo}{Dong-Chul Seo} & \href{https://www.daniellekilgo.com/}{Danielle Kilgo}  \\
%  & School of Public Health &  Hubbard School of Journalism and Mass Communication \\
%  & Indiana University Bloomington &  University of Minnesota \\
%  & \small{\href{mailto:seo@indiana.edu}{seo@indiana.edu},+1\,(812)\,855-9379} & \small{\href{mailto:dkilgo@umn.edu}{dkilgo@umn.edu}} \\
% \end{tabular}

\end{document}----------------------------